\documentclass[12pt]{article}

\usepackage{times}
\usepackage{textcomp}
\usepackage{listings}
\usepackage{fullpage}
\usepackage{color}
\usepackage{hyperref} 
\usepackage{pst-tree} 
\usepackage{verbatim} 
\usepackage{amsmath,amsfonts,amssymb,amsthm}

\def\part#1{\item[\bf #1)]}
\renewcommand{\thesubsection}{Question \arabic{subsection}}

\author{Clement Tsang}

\begin{document}

\begin{center}
\Large\textbf{CS 241, Lecture 5 - Assembler and Formal Languages}
\end{center}

\section{The Assembler}
\begin{itemize}
	\item We wish to input assembly code and output machine code.
    \item This translation process involves two phases: Analysis (understand what is meant by the input source) and Synthesis (output the equivalent target code in the new format).
    \item An assembly file is just a fuckton of characters.
    \item So in order to parse it correctly, we need to write a tokenizer that understands our tokens.
    \item Examples could be .word, MIPS instructions, labels, numbers, etc.  
    \item This is already done for us - we don't have to write a tokenizer (for now at least, thank god)
    \item Thus, all we have to really do is group tokens into instructions in our analysis, then in synthesis, output code.
    \item Assembler problem: Assume we have the following code:
        \begin{lstlisting}[mathescape, numbers=left, breaklines=true]
beq $\$$0, $\$$1, myLabel
myLabel:
    add $\$$1, $\$$1, $\$$1
        \end{lstlisting}
    \item The issue is that when we see myLabel for the first time, we don't know what the correct address is.
    \item What we'll do is that we'll go through the code twice - once to store all addresses of labelled instructions, second time when a label is referred to, we'll then look up the associated address on our symbol table/map and actually translate instructions to machine code.
    \item For us, we'll output our symbol table to stderr.
    \item This also means if we have a branch instruction we have to manually calculate the hard value of lines we jump by on our second iteration if it had a label as its third parameter.
\end{itemize}

\section{Binary in C++}
\begin{itemize}
    \item Recall we have three types of instruction formats - R, I, J.  
    \item R-format follows the following format: 
        \begin{itemize}
            \item opcode - 6 bits
            \item rs - 5 bits
            \item rt - 5 bits
            \item rd - 5 bits
            \item shift(shamt) - 5 bits
            \item funct - 6 bits
        \end{itemize}
    \item I-format follows the following format:
        \begin{itemize}
            \item opcode - 6 bits
            \item rs - 5 bits
            \item rt - 5 bits
            \item immediate - 16 bits, in twos complement usually
        \end{itemize} 
    \item J-format follows the following format:
        \begin{itemize}
            \item opcode - 6 bits
            \item psuedo-address - 26 bits
        \end{itemize}
    \item bne and beq are in I-format (has an immediate value), for example.  Let's use this for the following example.
    \item We can store an instruction as a single 32-bit integer, where we bitshift it the correct number of bits left, and bitwise or to join all the required values.
    \item For example, bne \$2, \$0, -3, would correspond to (5 $<<$ 26) $|$ (2 $<<$ 21) $|$ (0 $<<$ 16) $|$ offset
    \item An issue with the offset for an I-format is that the offset could be a 32-bit integer, but our offset is limited to a 16-bit integer.  
    \item This could cause a lot of problems with negative numbers, as we aren't shifting it or anything - just using a bitwise or will cause the offset to overwrite. 
    \item In order to solve this we'll use a mask to only get the last 16 bits.  We can do this by using a bitwise and with 0x0000ffff to forcefully zero out 4 of the 8 bytes.
    \item Another problem to overcome is that C++ will see our final result as just this big 9-byte integer.  This is NOT what we want - we want the binary!
    \item We want to do what happens with MIPS - when we sw to print, we print the LEAST SIGNIFICANT BYTE.
    \item Thus!  We will have to do MORE bit shifting, where we shift the bytes we want to print by the correct number of spots, in order to make them the LSBytes, then print them as an unsigned char.
\end{itemize}

\section{Formal Languages}
\begin{itemize}
    \item An \textbf{alphabet} is a \underline{non-empty} finite set of symbols often denoted by $\sum$.
    \item A \textbf{string/word} $w$ is a finite sequence of symbols denoted from $\sum$.  The set of all strings over an alphabet $\sum$ is denoted by $\sum*$.
    \item A \textbf{language} is a set of strings.
    \item The \textbf{length of a string} $w$ is denoted by $|w|$.
    \item For example, we can define $\sum = \{a, b, c,..., z\}$ to be our \textbf{normal English alphabet}.
    \item We could define $\epsilon$ to be the \textbf{empty string}, it is in $\sum*$ for any $\sum$ and $|\epsilon| = 0$.  
    \item For $\sum = \{0, 1\}$, \textbf{strings} include $w = 011101$, or $x = 1111$.  $|w| = 6$ and $|x| = 4$.
    \item A \textbf{language} could be $L = \{\}$ (the empty language), $L = \{ab^na : n \in \mathbb{N}\}$, which would be the set of strings over the alphabet $\sum = \{a, b\}$ consisting of an $a$ followed by 0 or more $b$ characters followed by an $a$.
    \item Another language example could be that for alphabet \{., -\}, $L$ = \{words in Morse Code\}.
    \item Our objective is that given a language, we want to determine if a string belongs to it (membership).
    \item The difficulty depends on the complexity of the language, from very easy to impossible.  In order of relative difficulty:
        \begin{itemize}
            \item finite
            \item regular
            \item context-free
            \item context-sensitive
            \item recursive
            \item impossible languages
        \end{itemize}
    \item For a finite language, membership is very easy as we just need to check for equality with every single word in the language.  How you do this search would affect efficiency but it is definitely doable.
\end{itemize}


\end{document}

