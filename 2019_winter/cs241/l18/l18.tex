\documentclass[12pt]{article}

\usepackage{times}
\usepackage{textcomp}
\usepackage{listings}
\usepackage{fullpage}
\usepackage{color}
\usepackage{hyperref} 
\usepackage{pst-tree} 
\usepackage{verbatim} 
\usepackage{graphicx}
\usepackage{amsmath,amsfonts,amssymb,amsthm}
\graphicspath{{./}}
\usepackage{courier}

\lstset{language=C, keywordstyle={\bfseries \color{red}}, basicstyle=\footnotesize\ttfamily}

\author{Clement Tsang}

\begin{document}

\begin{center}
    \Large\textbf{CS 241, Lecture 18: Code Generation (Continued)}
\end{center}


\section{MERL, IF, and WHILE}
\begin{itemize}
    \item They covered this today in lecture 18, but we covered it in lecture 17 because I thought they were faster.
    \item Note that I totally missed both lectures$\dots$ sleep is nice.  I could be missing some info.
\end{itemize}

\section{Pointers}
\begin{itemize}
    \item We need to support all of the following for pointers:
        \begin{itemize}
            \item NULL
            \item Dereferencing
            \item Address of
            \item Comparisons
            \item Pointer arithmetic
            \item Allocating and deallocating heap memory
            \item Pointer comparisons
            \item Pointer assignments and pointer access
        \end{itemize}

        \subsection{NULL}
        \begin{itemize}
            \item You would \emph{think} that we would make NULL 0x0.  But this is not valid - 0x0 is a valid memory address!
            \item We would like NULL to crash if we dereference (which will not happen with 0x0), so we pick a value for NULL that is not word-aligned --- not a multiple of 4!  Thus, we pick the value of 0x1.
            \item The code generation is as follows:
\begin{lstlisting}[mathescape, numbers=none, breaklines=true]
factor $\rightarrow$ NULL
code(factor) = add s3, s0, s11
\end{lstlisting}
            \item Note that attempting to use NULL with lw or sw will crash since MIPS is expecting a word-aligned address.
        \end{itemize}

        \subsection{Dereferencing}
        \begin{itemize}
            \item Consider code like \lstinline[mathescape]{factor1 $\rightarrow$ STAR factor2}.  The value in \lstinline[mathescape]{factor2} is a pointer (otherwise there's a type error, as per A8).
            \item What we want to do is access the value at \lstinline[mathescape]{factor2} and load it somewhere, in this case, \$3.  
            \item Since \lstinline[mathescape]{factor2} is a memory address, we want to load the value in the memory address \emph{at} \$3 and store it in \$3.
            \item The code generation is as follows:
\begin{lstlisting}[mathescape, numbers=none, breaklines=true]
code(factor1) = code(factor2) + lw s3, 0(s3)
\end{lstlisting}
        \end{itemize}
           
        \subsection{Addresss of}
        \begin{itemize}
            \item Recall an lvalue is something that can appear as the LHS of an assignment rule.
            \item Note that we have a rule \lstinline[mathescape]{factor $\rightarrow$ AMP lvalue}.
            \item Why this over \lstinline[mathescape]{factor1 $\rightarrow$ AMP factor2}?  Well, using \lstinline[mathescape]{factor2} gives us issues if we try to dereference an integer constant, which is a valid factor!
            \item So, when we have an lvalue, how do we find where it is in memory?  Symbol table!
            \item The code generation is as follows:
\begin{lstlisting}[mathescape, numbers=none, breaklines=true]
factor $\rightarrow$ AMP lvalue
code(factor) = lis s3 +
               .word offset +
               add s3, s3, s29
\end{lstlisting}
            where \lstinline[mathescape]{offset} is the offset found in the symbol table for the lvalue.
            \item What if in \lstinline[mathescape]{factor $\rightarrow$ AMP lvalue}, the lvalue is actually a \lstinline[mathescape]{STAR expr}?  That is, suppose we said \lstinline[mathescape]{&*ptr} in C.
            \item Well, we can just say that \lstinline[mathescape]{code(factor) = code(expr)}.
            \item Recall we had code for:
\begin{lstlisting}[mathescape, numbers=none, breaklines=true]
statement $\rightarrow$ lvalue BECOMES expr SEMI:

code(statement) = code(expr) + ;s3 $\leftarrow$ expr
                  sw s3, offset(s29)
\end{lstlisting}
            where \lstinline[mathescape]{lvalue} is an ID and \lstinline[mathescape]{offset} is the offset for the ID that is lvalue.
            \item How can we modify this if lvalue is of the form \lstinline[mathescape]{STAR factor}?
\begin{lstlisting}[mathescape, numbers=none, breaklines=true]
code(statement) = code(expr) + 
                  push(s3) + 
                  code(lvalue) + 
                  pop(s5) + ;s5 $\leftarrow$ expr
                  sw s5, 0(s3)
\end{lstlisting}
        \end{itemize}
       
        \subsection{Comparisons}
        \begin{itemize}
            \item The same as integer comparisons with one exception --- pointers cannot be negative and so, \lstinline[mathescape]{slt} is not what we want to use, but rather, \lstinline[mathescape]{sltu} (the thing that we never really used).
            \item But given \lstinline[mathescape]{test $\rightarrow$ expr COMP expr}, do we use \lstinline[mathescape]{slt} or \lstinline[mathescape]{sltu}?
            \item Just check the type of the first \lstinline[mathescape]{expr}; by Assignment 8 the two must be the same or else we would have failed the code!
            \item Tip: Augment the tree node class to include the type of the node itself if it has one.  You could also use the symbol table.
        \end{itemize}
       
        \subsection{Pointer Arithmetic}
        \begin{itemize}
            \item Recall for addition and subtraction, we have several contracts.
            \item For \lstinline[mathescape]{int + int} or \lstinline[mathescape]{int - int}, we proceed as before - but there are four more contracts that use pointers!
            \item For \lstinline[mathescape]{int* + int}:
\begin{lstlisting}[mathescape, numbers=none, breaklines=true]
expr1 $\rightarrow$ expr2 + term:

code(expr1) = code(expr2) +
              push(s3) +
              code(term) +
              mult s3, s4 +
              mflo s3 +
              pop(s5) +  ;s5 <- expr
              add s3, s5, s3
\end{lstlisting}
            Recall we are comuting a different memory address equal to \lstinline[mathescape]{expr2 + 4 $\times$ term}.
            \item For \lstinline[mathescape]{int + int*}:
\begin{lstlisting}[mathescape, numbers=none, breaklines=true]
expr1 $\rightarrow$ expr2 + term:

code(expr1) = code(expr2) + 
              mult s3, s4 +
              mflow s3 +
              push(s3) +
              code(term) + 
              pop(s5) +    ;s5 <- expr
              add s3, s5, s3
\end{lstlisting}
            Which corresponds to \lstinline[mathescape]{4 $\times$ expr2 + term}
            \item For \lstinline[mathescape]{int* - int}:
\begin{lstlisting}[mathescape, numbers=none, breaklines=true]
expr1 $\rightarrow$ expr2 - term:

code(expr1) = code(expr2) +
              push(s3) +
              code(Term) +
              mult s3, s4 +
              mflow s3 +
              pop(s5) +    ;s5 <- expr
              sub s3, s5, s3
\end{lstlisting}
            Which is just \lstinline[mathescape]{expr2 - 4 $\times$ term}
            \item For \lstinline[mathescape]{int* - int*}:
\begin{lstlisting}[mathescape, numbers=none, breaklines=true]
expr1 $\rightarrow$ expr2 - term:

code(expr1) = code(expr2) + 
              push(s3) +
              code(term) +
              pop(s5) +     ;s5 <- expr
              sub s3, s5, s3 +
              div s3, s4 +
              mflo s3
\end{lstlisting}
            Which is just \lstinline[mathescape]{(expr2 - term) / 4}
        \end{itemize}
        
        \subsection{Memory Allocation}
        \begin{itemize}
            \item Lastly, we need to handle the commands \lstinline[mathescape]{new, delete}
            \item Thankfully, we outsource this to a library, called \lstinline[mathescape]{alloc.merl}
            \item This \emph{must} be linked last in our output!  That is:
\begin{lstlisting}[mathescape, numbers=none, breaklines=true]
./wlp4gen < source.wlp4i > source.asm
cs241.linkasm < source.asm > source.merl
linker source.merl print.merl alloc.merl > exec.mips
\end{lstlisting}
            \item Our prologue now has the imports:
                \begin{itemize}
                    \item \lstinline[mathescape]{.import init}
                    \item \lstinline[mathescape]{.import new}
                    \item \lstinline[mathescape]{.import delete}
                \end{itemize}
            \item The command \lstinline[mathescape]{init} initalizes the heap, and must be called at the beginning!
            \item \lstinline[mathescape]{new} finds the number of new words needed as specified in \$1.
            \item It will return a pointer to memory at the beginning of this many words in \$3 if successful.  Otherwise, it places 0 in \$3.
            \item The code looks a bit like:
\begin{lstlisting}[mathescape, numbers=none, breaklines=true]
code(new int [expr]) = code(expr) +
                       add s1, s3, s0 +
                       call(new) +
                       bne s3, s0, 1 +
                       add s3, s11, s0
\end{lstlisting}
            Note that the last line sets \$3 to \lstinline[mathescape]{NULL} and executes iff the call to new fails!
            \item For \lstinline[mathescape]{delete}, it requires that \$1 is a memory address to be deallocated.
            \item The code looks like:
\begin{lstlisting}[mathescape, numbers=none, breaklines=true]
code(delete [] expr) = code(expr) +
                       beq s3, s11, skipDelete: +
                       add s1, s3, s0 +
                       call(delete) +
                       skipDelete:
\end{lstlisting}
            Again, like \lstinline[mathescape]{if, while}, we need to count the deletion labels.  We skip delete if attempting to delete a null pointer.
    \end{itemize}
\end{itemize}

\end{document}

