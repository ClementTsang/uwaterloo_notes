\documentclass[12pt]{article}

\usepackage{times}
\usepackage{textcomp}
\usepackage{listings}

\author{Clement Tsang}

\begin{document}

\begin{center}
\Large\textbf{CS 241, Lecture 3 - More on Machine Language}
\end{center}

\section{Addition examples}
\begin{itemize}
    \item Add 11 and 13, return in register 3.
        \begin{lstlisting}[mathescape, numbers=left, breaklines=true]
lis $\$$1
.word 11
lis $\$$2
.word 13
add $\$$3, $\$$1, $\$$2
jr $\$$31
        \end{lstlisting}
\end{itemize}

\section{Multplication and Division}
\begin{itemize}
    \item We get a problem with multiplying and division - we may need more space when multiplying (ie: $2^{30} \times 2^{30} = 2^{60}$), and when dividing, we want the quotient and remainder!
    \item For multiplication, we COMBINE the hi and lo registers to get a 64-bit register.  The most significant word is placed in hi, and least significant word in lo (most sig word is largest 4 bytes).
    \item \lstinline[mathescape]{mult $\$$s, $\$$t}.
    \item For division, we put the quotient \lstinline[mathescape]{$\$$s $\div$ $\$$t} in lo and the remainder \lstinline[mathescape]{$\$$s $\%$ $\$$t} in hi.  Preforms s / t
    \item To access data from hi and lo, we use the \lstinline[mathescape]{mfhi $\$$d} and \lstinline[mathescape]{mflo $\$$d} instructions to move from the hi/lo register to the register we state.
\end{itemize}

\section{RAM}
\begin{itemize}
    \item RAM uses DRAM which is cheaper but slower than SRAM (so slower than register but more storage than SRAM due to it being cheaper).
    \item Connect RAM and CPU via bus.
    \item Starts from address 0 and increments by the word size (4).
    \item We cannot use data straight from RAM, must transfer to registers first before operations can be done.
    \item We use the command \lstinline[mathescape]{lw $\$$t, i($\$$s)} to load the word in \$s + i and stores it in the memory address \$t.  
    \item i is an immediate; it is an integer, twos-complement number converted to binary.
    \item We use \lstinline[mathescape]{sw} to do the opposite.
    \item For example, suppose \$1 contains the address of an array and \$2 contains the number of elements in the array.  Place the immediate value 7 in  the last possible spot in the array.
        \begin{lstlisting}[mathescape, numbers=left, breaklines=true]
lis $\$$3
.word 4
mult $\$$3, $\$$2
lis $\$$4
.word 7
mflo $\$$6
sw $4 -4($\$$6) 
        \end{lstlisting} 
\end{itemize}

\section{Branching}
\begin{itemize}
    \item Use the \lstinline[mathescape]{beq $\$$s, $\$$t, i} or \lstinline[mathescape]{bne} (same arguments) to branch if the two registers are equal or not equal, respectively.
    \item We skip our PC by $i \cdot 4$ bytes, or i words, forward.
    \item For example, place the value 1 in register 2 if register 1 is even, and place 0 in register 2 otherwise.
    \begin{lstlisting}[mathescape, numbers=left, breaklines=true]
lis $\$$3
.word 1
lis $\$$4
.word 2
div $\$$1 $\$$5
mfhi $\$$4
add $\$$2, $\$$0, $\$$0
beq $\$$0, $\$$4, 1 
add $\$$2, $\$$0, $\$$3
jr $\$$31
    \end{lstlisting} 
    \item Note that branching assumes you already incremented by 4.  Remember 251 - we increment by 4 in parallel with the instruction when it goes into the instruction memory stage.
    \item \lstinline[mathescape]{slt d s t} instruction will set d to be 1 if s < t, if s >= t then it sets it to be 0.
    \item If we were to redo the above example using just slt (got lazy, didn't add register values, doesn't matter as no immediate values):
        \begin{lstlisting}[mathescape, numbers=left, breaklines=true]
lis 3
.word 1
lis 4
.word 2
div 1 5
mfhi 4
slt 2 3 4 ; if 4 >= 3, then 4 is 1, so set 2 to 0.  Othwerise, if 4 < 3, then 4 must be a 0 (even) and so set 2 to 1.
jr 31
        \end{lstlisting} 
    \item Example: write a program that negates the value in reg 1 if it is positive.
        \begin{lstlisting}[mathescape, numbers=left, breaklines=true]
slt 2 1 0
bne 2 0 1
sub 1 0 1
jr 31
        \end{lstlisting}
    \item  
\end{itemize}

\subsection{Looping}
\begin{itemize}
    \item Example: add all even numbers from 1 to 20, store in register 3.
        \begin{lstlisting}[mathescape, numbers=left, breaklines=true]
lis 1
.word 0x20
lis 2
.word 0x2
add 3 0 0
add 3 3 1
sub 1 1 2
beq 1 0 -3
jr 31
        \end{lstlisting}
    \item We can use labels to make the loops less awful and more readable:
        \begin{lstlisting}[mathescape, numbers=left, breaklines=true]
lis 2
.word 20
lis 1
add 3, 0, 0
top:
    add 3, 3, 2
    sub 2, 2, 1
    bne 2, 0, top
    jr 31
        \end{lstlisting}        
\end{itemize}


\end{document}

