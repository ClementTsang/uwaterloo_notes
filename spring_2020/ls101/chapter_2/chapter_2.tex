\documentclass{article}

\usepackage{times}
\usepackage{textcomp}
\usepackage{listings}
\usepackage{fullpage}
\usepackage{color}
\usepackage{courier}
\usepackage{verbatim}
\usepackage{graphicx}
\usepackage{amsmath, amsfonts, amssymb, amsthm}
\usepackage{hyperref}
\graphicspath{{./}}

\lstset{language=python, keywordstyle={\bfseries \color{green}}, basicstyle=\footnotesize\ttfamily}
\setlength{\paperheight}{11in}
\author{Clement Tsang}

\begin{document}

\begin{center}
    \Large{LS 101 --- Chapter 2: Legal Ethics and the Law Profession}
\end{center}

\section{Becoming a Lawyer}
\begin{itemize}
    \item A law degree is an undergraduate program, which often requires graduation from some other undergraduate degree.
    \item Becoming a lawyer requires a law degree, a year spent articling, and completion of the bar admission exams.
    \item In Ontario, bar exams are done through the Provincial Law Society of Upper Canada.
    \item Somewhat dominated by men, though starting to change.  Same with ethnicity.
    \item Articling means you are working with lawyers under their supervision to learn from more experienced mentors.  Typically for a year.
    \item Must take an oath to uphold moral integrity, dating back to the Roman Theodosian code and the Anglo-Saxon code.
    \item Applicants to the bar must be of good moral character.
    \item There is minimal screening of applicants based on moral character.
    \item Often rare for someone to be denied for this though.  Also not very consistent; very subjective and is often dealt with in a case-by-case basis.
    \item Usually requires someone to bring forward evidence to why one should deny an applicant.
    \item The Pardons Act allows one to get a pardon after 2 years of good behaviour, and in an indictable offence, after 5 years.  This gives a chance for people to rehabilitate themselves when dealing with this.
    \item Sometimes misused; denying people that others just didn't like.
    \item Also rare to disbar a lawyer.  Typically happens due to how they practice, or sometimes for other reasons.
\end{itemize}

\section{Legal Ethics}
\begin{itemize}
    \item The law profession is both admired and disliked.
    \item The ethics of lawyers is often of dispute.
    \item For example, the stereotype of some lawyers is to be greedy, lying, cheating, etc.
    \item But at the same time, they often take on noble causes that fight against the bigger parties like governments, companies, etc.
    \item Sometimes feels like lawyers are willing to fight anyone.
    \item Lawyers are often placed in a situation where they have to fight ethical dilemmas; they may be told to defend someone who is not exactly the greatest of people.
    \item Often also frequently involved in corruptions and scandals.
    \item May be tainted by criminals.
    \item There is little training in law on how to deal with ethical issues.
\end{itemize}

\section{Social Control and the Legal Profession}
\begin{itemize}
    \item Law is an autonomous profession that is largely self regulated.
    \item Since law is so hard to get into and one cannot perform law unless credited, this somewhat excludes laymen from engaging in law and has made it almost a monopoly.
    \item Paralegals are a more recent(?) role that is somewhat more in the middle of the law and normal people.
    \item This also does protect the profession to having some standards, so it's both good and bad.
    \item Law societies in each province set rules of conduct and deal with disciplinary problems.
    \item Lawyers are subject to legal controls and informal social controls within the profession.
\end{itemize}

\section{The Canadian Bar Association Code of Professional Conduct}
\begin{itemize}
    \item This code offers a guideline for professional conduct for all of Canada.
    \item Lawyers are often seen as protecting the public trust and ensure that the legal system is respected.
    \item As such, lawyers are supposed to act ethically to ensure the civil legal system does not fall into disrepute.
    \item Otherwise, if people don't have faith in the system, they will not want to use it!
    \item Often criticized for being relatively vague.
\end{itemize}

\section{The advocate as a Hired Hand}
\begin{itemize}
    \item The traditional role of the lawyer is that of a hired hand --- someone hired to act as an advocate on behalf of their client.
    \item This gives lawyers a bit of an identity.
    \item Some, however, see lawyers as hired guns (adversarial).
    \item This also gives rise to the idea of lawyers being value-neutral; and as such also justifies that they are neutral to whatever their client has actually done.
    \item That is, they are being paid to help, and that is their job regardless.  They should be impartial, that is their job.
\end{itemize}

\section{The Adversarial System}
\begin{itemize}
    \item In the adversary model, lawyers are on each side in the court room and fighting for their clients.
    \item Based in trials in ordeal/battle where people would fight physically believing that justice would be served and the right would beat the wrong.
    \item Now, it's just that but with lawyers.
    \item Critics argue that this model obscures the truth and benefits the wealthy.  It can be very costly to use the judicial system and thus it favours the rich.
    \item For example, lawyers helping tobacco companies despite knowing fully well it is harmful, tax lawyers advising clients on how to dodge taxes.
    \item One could say that lawyers are just following the hired hand model though; they are just entitled to act in this way, even if it results in injustice, as that is just the lawyer's job.  They are to work for their client and are not responsible, and are neutral to moral responsibilities.
    \item Even if someone guilty goes free, this is just part of the system.
    \item Critics argue that the client's interests \emph{should} take precedence, but they should also consider other things; but one could counter this by saying that one would expect a lawyer to defend the client to the best of their ability.
    \item Lawyers are not allowed to mislead courts, but this gets hazy with \emph{omitting} evidence; this is an ethical issue.
\end{itemize}

\section{Ethical Norms when Dealing with Clients}
\begin{itemize}
    \item An attorney has many obligations when they take on a client.
    \item Cannot discriminate based on race/ethnicity/religion.
    \item Often may take on unpopular cases/causes/clients.
    \item Should defend a client competently.
    \item Obligations to clients include:
        \begin{itemize}
            \item Partisanship --- they must act on behalf of their client and be committed.
            \item Competence --- lawyers must deliver service in a competent, diligent manner as an expert.
            \item Loyalty --- they should place the client's interest above all else, nor should they get into a conflict of interest, and disclose if any will arise or exist (can't accept a referral fee, cannot represent against a previous client if it is related; firms in general may forbidden from acting against previous clients due to this as well, and even if it's just consulting).
            \item Candour --- lawyers should be open and honest, give candid advice, the probably outcomes, costs, being reasonable with expectations.
            \item Proactivity --- Should be proactive rather than reactive; anticipate rather than wait, within limits (don't get too zeal).
            \item Confidentiality --- don't divulge information that cannot be divulged.  For example, if a person is seeking legal advice before turning themselves in, lawyers should not divulge this.
        \end{itemize}
\end{itemize}

\section{Lawyer and Client Conflict}
\begin{itemize}
    \item Clients can dismiss their attorneys for any reason.
    \item Clients can also sue their lawyer for negligence or other causes.  This must be a mistake that a reasonably competent lawyer would not make.
    \item Judges can step in and make lawyers refund their costs, charge a lawyer with contempt, award costs to another party, disallow costs between a lawyer and client, etc.
    \item Lawyers can only withdraw their services for specific reasons --- for example, clients obstruct them, being asked to do something they don't wish to do, if they realize they are no longer competent in the situation, the client not cooperating, the client not paying their bills, etc.
    \item These are not litigious cases though (have not gone to court).
    \item If the lawyer is already defending, then they would have to go through the court to deal with withdrawing from the case.
\end{itemize}

\section{The Paul Bernardo Case}
\begin{itemize}
    \item Ken Murray, Bernardo's first lawyer, removed incriminating tapes from Bernardo's home.
    \item Murray later handed the tapes over to Bernardo's new lawyer 17 months later.
    \item Murray was charged with obstructing justice.  Murray pleaded that he had to wait before turning them in and that he had no legal rights to hand them in to the police, as well as client confidentiality.
    \item However, he had technically tampered with evidence by removing them from the home.
    \item They also ruled while the conversations were confidential, he could not use client privilege in regards to evidence.
    \item They claim that once Murray had learned of their value as evidence, he should have turned them over immediately.
    \item However, they also acquitted him on that he did not intend to do this with criminal intent; and the official reason for acquitting was that Murray intended to use them as a defence against another person.
    \item Skeptical but accepted with benefit of the doubt.
    \item The Law Society of Upper Canada also withdrew charges of professional misconduct since the ethical guidelines were very vague in this situation, and that even if he had asked for help, it might have just led to confusion.
    \item Now, it is fully defined that doing so is an offence and will be seen as obstructing justice.  So is taking evidence from the scene, or destroying it, or hiding it.
\end{itemize}

\section{The Role of the Crown Attorney}
\begin{itemize}
    \item The Crown Attorney is typically a lawyer who plays the role of prosecutor in trials.
    \item They are government employees with salaries.
    \item If they believe that there is enough evidence to lead to a conviction, then they proceed with a case; otherwise they'll likely drop the case.
    \item The Crown Attorney represents the public interest.
    \item The Crown is still expected to be fair and impartial.  They aren't looking for a conviction, like a lawyer.  They want the truth.
    \item Their job is to ensure that due process occurs, and present evidence such that the truth is revealed.
    \item Plea bargaining occurs in many cases and involves both legal and moral judgements.
\end{itemize}

\section{A Conflict Perspective Critique on the Legal System}
\begin{itemize}
    \item Conflict theorists view the legal system as supporting the wealthy.
    \item The rich use the legal system to maintain their privileged position.
    \item The poor often do not have access to the legal system because of the costs.
\end{itemize}

\section{A Lawyer who Murders}
\begin{itemize}
    \item Lawyer who murdered his wife and sentenced to life imprisonment.
    \item Disbarred by the Law Society of Alberta in 1990.
    \item Paroled from prison and applied to the bar in 1998.
    \item Interesting case as he was disbarred for something outside of the legal practice.
    \item Sychuk was denied permission by the argument of him being permitted to repractice law would impact the status of the legal profession.
    \item Despite this, another murderer from Quebec, Sebastian Brousseau, was allowed to enter the bar admission program after being released --- his case was argued to be okay after being initially denied twice, as it was seen as him being young and stupid (was 19).
    \item Meanwhile, Sychuk was older and could not be given the same treatment.
\end{itemize}

\end{document}
