\documentclass{article}

\usepackage{times}
\usepackage{textcomp}
\usepackage{listings}
\usepackage{fullpage}
\usepackage{color}
\usepackage{courier}
\usepackage{verbatim}
\usepackage{graphicx}
\usepackage{amsmath, amsfonts, amssymb, amsthm}
\usepackage{hyperref}
\graphicspath{{./}}

\lstset{language=c, keywordstyle={\bfseries \color{red}}, basicstyle=\footnotesize\ttfamily}
\setlength{\paperheight}{11in}
\author{Clement Tsang}

\begin{document}

\begin{center}
    \Large{CS 458 --- Module 4: Networks}
\end{center}

\section{Intro to Networks}
\begin{itemize}
    \item To create a network, you need 3 things:
        \begin{enumerate}
            \item Devices able to receive and send signals
            \item A way to connect devices to each other
            \item Rules for communicating, or a protocol
        \end{enumerate}
    \item Some examples of protocols:
        \begin{itemize}
            \item Token ring --- a person can only talk if they have the token
            \item CSMA/CD --- all listen to the wire, if they hear no signal they try to transmit.  If there is a collision, then they all stop and resend.
        \end{itemize}
    \item Well what are some problems?
        \begin{itemize}
            \item The Internet's design connects many computer networks together.  It also assumes that participants are honest and will cooperate --- they will not look at messages that don't belong to them, they will not delete your messages, etc.  Everyone should mutually work together\dots right?
            \item There's also no routing logic in the addressing scheme --- given some IP address, who knows where it comes from?  For example, a phone number has an area/country code.  An IPV4 address like \lstinline{136.192.63.0} could come from anywhere!
            \item Nor can you control the path your message follows!
            \item Your message can be broken up with each part following a different route.
            \item There is no real hard stop limit to the number of nodes (at least everywhere).
            \item It's really hard to conceptualize.
            \item Nobody is in charge (both good and bad).
        \end{itemize}
\end{itemize}

\section{Daemons, Servers, Ports}
\begin{itemize}
    \item A server is a computer on a network to do tasks for other computers (clients).
    \item A daemon is like a servant that can only do one task within a server.
    \item We can think of a server like a huge apartment building, and each apartment can have one servant (daemon).
    \item For example, the mail sending daemon (SMTP) is 25.
    \item Some apartments (ports) can be empty.  Many ports are actually empty!
    \item One could hide a service in a port it's not supposed to be in.
    \item For example, an HTTP daemon is in port 80.  This is implied by default (ie: \lstinline{https://www.uwaterloo.ca} implies \lstinline{https://www.uwaterloo.ca:80}).
    \item But one could put a web service at, for example, port 8080.
    \item A ``loose-lipped'' system may reply to an attacker and advertise what services they are running \emph{and} what at what port.
\end{itemize}

\section{Port Scanning, Information Gathering, Wiretapping, Impersonation}

\subsection{Port Scanning}
\begin{itemize}
    \item A port scan checks every port in sequence.
    \item We would ideally want, at least for security, to not reply when ports are checked.
    \item Unfortunately, this isn't really possible as we need this replying for actual use.
    \item Tools like \lstinline{nmap} would give many details about a machine.
    \item A command like \lstinline{finger} allows you to look up a user in a machine.  If this is not closed, one could do it from outside a machine\dots
    \item But this is just the beginning\dots maybe you could get some info, but port scanning is not really malicious on its own yet.
\end{itemize}

\subsection{Intelligence Gathering}
\begin{itemize}
    \item Social Engineering is attacking people via exploiting other humans, which can give valuable information.
    \item Pretending to be part of an organization they're not, exploiting the helpful nature of people (``I forgot my password''), distractions to grab information somehow, etc.
    \item Other ways you could get info?
        \begin{itemize}
            \item Dumpster diving
            \item Eavesdropping
            \item Lots of things placed online that shouldn't be there --- Google, social media, etc.
        \end{itemize}
    \item Wiretapping
        \begin{itemize}
            \item Two types: passive and active.
            \item Passive wiretapping is basically just eavesdropping.  When a message is sent, a node could read the destination data --- but there is \emph{nothing} stopping Eve from looking at the data!
            \item The analogy is an envelope with an address and a non-sealed back.
            \item Active wiretapping will require modification/fabrication of communication.
            \item For example, Mallory could modify a message sending money from one account to another.  That is, Mallory is usually a MITM during an active wiretap attack.
            \item One can also eavesdrop while communication is flowing through a link; we call ``promiscuous sniffing''.
            \item We should \emph{always} assume someone is eavesdropping the data!
            \item The degree of vulnerability would depend on the communication media:
                \begin{itemize}
                    \item For example, copper cables mean that a physically close attacker could eavesdrop without making physical contact, or just cut the cable open/splice in another cable.
                    \item Coaxial cables help shield some of this signal from leaking out compared to twisted pair cables.
                    \item Optical fibre would be harder, as there is no inductance and signal loss caused by splicing would be noticeable.
                    \item Unbound transmission is through the air --- WiFi, microwaves, radio, etc.
                    \item This is versus bounded, like cables.
                    \item How could we protect something like WiFi?  Problems are:
                        \begin{itemize}
                            \item It is easy to intercept with anything that can use WiFi.
                            \item It's easy to read packet info like destination and source IP addresses --- even at a distance!
                            \item Physical barriers are useless for a wireless network.
                            \item Wireless APs can also be faked; one could use a router that is not actually owned by the network you are connecting to to steal credentials.
                        \end{itemize}
                \end{itemize}
            \item When we transmit data, how do we choose what medium?
                \begin{itemize}
                    \item Is it sensitive?  If it is, we probably don't want to use an unbounded medium.
                    \item Are there \emph{segments} of the network carrying sensitive data?
                    \item Would one notice of an intruder is eavesdropping?  For example, using barriers that would make an attack obvious due to damage to said barrier.
                    \item Are backbone segments accessible?  Can an intruder actually attack said parts of the network?
                \end{itemize}
        \end{itemize}
\end{itemize}

\subsection{Impersonation}
\begin{itemize}
    \item A person could try to log into a machine that does not belong to them by pretending to be an owner.
    \item Steal passwords, guess, social engineering, sniff password, etc.
    \item Or pretend to act like a machine itself.
\end{itemize}

\end{document}
