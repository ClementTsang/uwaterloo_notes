\documentclass{article}

\usepackage{times}
\usepackage{textcomp}
\usepackage{listings}
\usepackage{fullpage}
\usepackage{color}
\usepackage{courier}
\usepackage{verbatim}
\usepackage{graphicx}
\usepackage{amsmath, amsfonts, amssymb, amsthm}
\usepackage{hyperref}
\graphicspath{{./}}

\lstset{language=python, keywordstyle={\bfseries \color{blue}}, basicstyle=\footnotesize\ttfamily}
\setlength{\paperheight}{11in}
\author{Clement Tsang}

\begin{document}

\begin{center}
    \Large{CS 486 --- Lecture 12: Bayesian Networks Continued}
\end{center}

\section{Bayes Networks}
\begin{itemize}
    \item A node is conditionally independent of its non-descendants given its parents.
    \item A Markov blanket of a node consists of its parents, its children, and its children's parents.
    \item A node is conditionally independent of all other nodes given its Markov blanket.
    \item Given a Bayesian network, how do we determine if two variables $X$ and $Y$ are independent if we observe the values of a set of variables $E$?
    \item We define \emph{d-separation} as: A set of variables $E$ d-separates $X$ and $Y$ if $E$ blocks every undirected path between $X$ and $Y$ in the network.
    \item If $E$ d-separates $X$ and $Y$, then $X$ and $Y$ are conditionally indep. given $E$.
    \item Three cases:
        \begin{itemize}
            \item $A \rightarrow E \rightarrow B$ --- if $E$ is observed then it blocks 
            \item $A \leftarrow E \rightarrow B$ --- if $E$ is observed then it blocks
            \item $A \rightarrow E \leftarrow B$ --- if $E$ \emph{and} its descendants are \emph{not} observed they block
        \end{itemize}
\end{itemize}

\section{Constructing Bayesian Networks}
\begin{itemize}
    \item For a joint probability distribution there are multiple correct Bayesian networks.
    \item We prefer one network over another if it requires fewer probabilities (smaller and easier to work with).
\end{itemize}

\end{document}
