\documentclass{article}

\usepackage{times}
\usepackage{textcomp}
\usepackage{listings}
\usepackage{fullpage}
\usepackage{color}
\usepackage{courier}
\usepackage{verbatim}
\usepackage{graphicx}
\usepackage{amsmath, amsfonts, amssymb, amsthm}
\usepackage{hyperref}
\graphicspath{{./}}

\lstset{language=python, keywordstyle={\bfseries \color{blue}}, basicstyle=\footnotesize\ttfamily}
\setlength{\paperheight}{11in}
\author{Clement Tsang}

\begin{document}

\begin{center}
    \Large{CS 486 --- Lecture 13: Variable Elimination Algorithms}
\end{center}

\section{VE Motivation}
\begin{itemize}
    \item For something like $P(B = b | W = t \wedge G = t), b \in {t, f}$, our shorthand is $P(B | w \wedge g)$. 
    \item This is equal to:
        \[
            P(B | w \wedge g) = \frac{P(B \wedge w \wedge g)}{P(b \wedge w \wedge g) + P(\neg b \wedge w \wedge g)}
        \]
    \item So for $P(B \wedge w \wedge g)$, we can say it's equal to:
        \[
            \sum_a \sum_e \sum_r P(B)P(e)P(r|e)P(a|B \wedge e) P(w|a)P(g|a)
        \]
        Such an expression would take 47 operations to do!
    \item Note we can simplify this:
        \[
            P(B) \sum_a P(w|a)P(g|a) \sum_e P(e)P(a | B \wedge e)
        \]
        Note the $P(r|e)$ term is removed as summing over $r$ means it must equal 1.
    \item This would take \emph{12} operations.
\end{itemize}

\section{VE Algorithm}
\begin{itemize}
    \item Use dynamic programming (do calculations once if possible) and exploit conditional independence to reduce the number of operations required.
    \item VEA relies on factors and operations on factors.
    \item A factor is a function from some random variables to a number.
    \item For example, $f(X_1, X_2)$ could be $P(X_1 \wedge X_2)$ or $P(X_1 | X_2)$.
    \item We define a factor for every conditional probability distribution in the Bayes net.
    \item We can \emph{restrict} a factor by assigning a value to the variable in the factor.
    \item If one restricts until all values have assigned values in the factor, it is now just a number (obviously).
    \item \emph{Sum out} operations sum out a variable:
        \[
            (\sum_{X_1} f)(X_2,\cdots,X_j) = f(X_1 = v_1,\dots,X_j) + \cdots + f(X_1=v_k,\dots,X_j)
        \]
    \item We can also \emph{multiply} two factors together.  The product of two factors $f_1(X, Y)$ and $f_2(Y, Z)$ where $Y$ is in common will give $(f_1 \times f_2)(X, Y, Z)$.
    \item \emph{Normalizing} divides each value by the sum of all the values.  This ensures that we have a valid probability distribution (ie: if we had values of 0.2 and 0.6, we would want 0.25 and 0.75 as probabilities).
    \item The general VE algorithm:
        \begin{enumerate}
            \item Construct a factor for each conditional probability.
            \item Restrict observed variables to their observed values.
            \item Eliminate each hidden variable via multiplying and summing out.
            \item Multiply remaining factors.
            \item Normalize the resulting factor.
        \end{enumerate}
\end{itemize}

\end{document}
