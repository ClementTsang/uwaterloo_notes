\documentclass{article}

\usepackage{times}
\usepackage{textcomp}
\usepackage{listings}
\usepackage{fullpage}
\usepackage{color}
\usepackage{courier}
\usepackage{verbatim}
\usepackage{graphicx}
\usepackage{amsmath, amsfonts, amssymb, amsthm}
\usepackage{hyperref}
\graphicspath{{./}}

\lstset{language=python, keywordstyle={\bfseries \color{green}}, basicstyle=\footnotesize\ttfamily}
\setlength{\paperheight}{11in}
\author{Clement Tsang}

\begin{document}

\begin{center}
    \Large{LS 101 --- Chapter 3: Theories of Social Order}
\end{center}

\section{The Consensus Model}
\begin{itemize}
    \item Social order can't be taken for granted; it is often argued that it is a social construct that one must work towards to achieve.
    \item For example, some countries experience social unrest and a lack of social order.
    \item The consensus model states that society is held together by common values --- what is important, what is good, what is bad, etc.
    \item Values are a blueprint.
    \item Common norms define what we see as good/bad, religious rules, customs, etc.
    \item These norms become formalized as society becomes larger, into laws.
    \item As such, the laws are just a formalized view of what is the society upholds.
    \item People learn the norms/values through society, and the laws are just an extension from this.
    \item Argues that people are satisfied by conforming to the norms of society.
    \item Law reflects the collective will of the people; they agree on the definition and what the law should be.
    \item Thus, this legitimizes the government and its power; that the laws the government passes are what the society wants.
    \item A criticism is that this requires a utopian sort of view where society is homogenous and all agree on the same ideals.
    \item But in this case, there would be no conflicts in the world as well, as everyone would act and value the same things.
    \item It explains crime/social disorder as the failure of society to properly socialize people, or they were socialized with values that conflict with those of society.
    \item As such, the criminal is the anomaly --- part of the system that hasn't worked ---, and to deal with the problem is to deal with the criminal.
    \item Rehabilitate or resocialize the criminal.
    \item Then, according to this model, the criminal should fit back into society.
\end{itemize}


\section{The Pluralistic/Interests Model}
\begin{itemize}
    \item Recognizes that societies are heterogeneous/pluralistic --- that people have different interests and values that may compete with others.
    \item Furthermore, while a socialistic model may have worked with a smaller society, it gets harder and harder to maintain as society grows --- for example, not every one is the same religion, not everyone is the same ethnic origin, etc.
    \item Conflict resolution mechanisms are developed through a consensus to resolve disputes/conflicts.
    \item For example, governments, the courts, tribunals, etc.
    \item The legal system should be value neutral.
    \item Laws are often developed through consensus.
    \item One criticism is that some lobbying groups are stronger --- this means that some laws that are passed are more biased to groups with more power.
\end{itemize}

\section{Conflict Resolution Through the Courts}
\begin{itemize}
    \item In 2008, the dioceses of Niagara were denied access to two parishes that voted to leave the Anglican church of Canada.
    \item Since the Anglican Church of Canada is accepting gay priests, the network is setting up a separate church as they feel that this is against their spiritual beliefs.
    \item Originally, the agreement was to split the usage of a church equally.
    \item They went to court, and in the original ruling, the Anglican network of Canada won, and the Anglican church of Canada don't have access to their former parishes.
    \item Similar case in Australia with Sikhs.  They had a developed a religious community, church, facilities, etc.
    \item An emissary was sent over from another country to take charge; this created some conflicts.
    \item Similar story; tried to break away, church claimed to own the property, but the parishioners disagreed as they were the ones who bought and built the property.
    \item This also went to court, and the parish ended up losing the case initially.
    \item When they appealed, it went to supreme court and the law went on the side of the church.
    \item They then attacked the church's representatives after the final ruling, which brought the opinion of the parishioners down as people saw that they had already had their day in court, despite the public supporting them initially.
\end{itemize}

\section{The Conflict/Coercion Model}
\begin{itemize}
    \item Emphasizes the repressive nature of government.
    \item Suggests that social order is developed and maintained through coercion.
    \item A small elite in society controls the economic and political power.
    \item Force and coercion are used to maintain the status quo.
    \item Law serves the vested interests of those in power.
    \item Examples of this view are Karl Marx's writings on society, where the bourgeoisie own the means of production, while the proletariats are the common people.
    \item Societies following this would often be very totalitarian, where there is little free speech or choice.
    \item Through force or fear, control is maintained.
    \item To people, there is no legitimacy to the law --- they do not believe in it, they only respect it due to fear.
    \item This means that these societies are often not stable and usually these societies end up being overthrown given the option as it is not in their interest to maintain the status quo.
    \item The law is meant to maintain the status quo.
    \item Basic principles:
        \begin{itemize}
            \item Society is made up of many groups; some groups often get priority.
            \item Society is pluralistic; as such the definitions of rights/wrongs, goals/interests, etc. will conflict.  A lid must be maintained via coercion.
            \item Imbalance in political power; vast majority of people will not have any.
            \item Laws are biased.  They serve the interests of the ruling class.  The courts will not be impartial or fair.  As such, the conflict resolution mechanisms don't work, so people don't want to use them.
            \item No faith in the police, the judicial system, etc.
            \item The people in power do not want to see any change.
        \end{itemize}
    \item Force is used instead of bargaining, compromise, etc.
    \item Some of the lower class are driven to commit crimes to survive.
\end{itemize}

\section{Overview of all Models}
\begin{itemize}
    \item All three models have some degree of validity, and different societies use these in various degrees.
    \item Totalitarian and authoritarian societies often depend on coercion; most democratic societies rely on consensus and conflict resolution mechanisms.
\end{itemize}

\end{document}
