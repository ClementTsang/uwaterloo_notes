\documentclass{article}

\usepackage{times}
\usepackage{textcomp}
\usepackage{listings}
\usepackage{fullpage}
\usepackage{color}
\usepackage{courier}
\usepackage{verbatim}
\usepackage{graphicx}
\usepackage{amsmath, amsfonts, amssymb, amsthm}
\usepackage{hyperref}
\graphicspath{{./}}

\lstset{language=python, keywordstyle={\bfseries \color{green}}, basicstyle=\footnotesize\ttfamily}
\setlength{\paperheight}{11in}
\author{Clement Tsang}

\begin{document}

\begin{center}
    \Large{LS 101 --- Chapter 1: Introduction to Legal Concepts}
\end{center}

\section{Overview}
\begin{itemize}
    \item Traditional societies regulate behaviour through informal norms and social controls; but this is not enough for modern complex societies.
    \item Things like religion, small community size, etc. were effective enough; this is becoming less and less relevant or possible.
    \item Laws are formal and explicit rules that are:
        \begin{enumerate}
            \item Established and formalized by a political body.
            \item Enforced by threats of punishment/penalties or coercion.
            \item Enforced by formal agencies of control (ex: police, courts).
        \end{enumerate}
    \item Without law, may things are just not possible, like business.
    \item Law can symbolize the power of the government as they can use power to control people.
    \item Lawyers are advocates who represent clients in legal conflicts.
    \item Lawyers seek advantages for their clients and are partial and \emph{not} objective, since they seek to win for their client rather than the truth.
    \item One could say lawyers are in an intermediate position; they are the intermediary between a client and the court.
    \item They enter the courtroom from a separate entrance and last, announced, have a separate chamber, highest chair, separated via barriers, etc.
    \item Symbolizes just how powerful a judge is in the role of law and court.
    \item Can also charge people within the court for contempt of court.
    \item Judges are the highest authority in the court system and are expected to be knowledgeable, fair, and impartial.
\end{itemize}

\section{Functions of the Law}
\begin{itemize}
    \item The law functions to maintain social control and ensure social order.
    \item Major mechanisms for dispute/conflict resolution; we may disagree with viewpoints but we can agree on how to settle disagreements/conflicts in a peaceful way.
    \item Religion can both support and undermine laws; sometimes the religious will see that religion is a higher, stronger law, whether that is accepted by the government or not.
    \item Also a mechanism for social change.  Taxation law, for example, legitimizes that wealthier people are taxed more, or thinks like smoking laws making smoking seen as a ``wrong'' thing.
    \item At the same time, it can prevent some social change/keep status quo; for example it may enforce how some things are done.
    \item Laws uphold rights/privileges.
    \item Laws determine duties/obligations.
    \item Laws communicate and uphold moral standards, upholding certain ideological values, educating people, and socializing people into these laws.  In a way, it's what the people of the place see as good/bad, where people who go against the law are going against a moral standard.
    \item Laws are criticized for preventing change, favouring the rich, promoting injustice/inequality, and are expensive/inefficient to enforce.
    \item An example of ``bad'' laws were Jim Crow laws, or apartheid laws.
    \item One example is how the NRA uses the 2A to prevent any real gun law changes.
\end{itemize}

\section{Mediation, Arbitration, Adjudication}
\begin{itemize}
    \item One way to resolve disputes.
    \item Involves a neutral third party who attempts to find a compromise between the disputants.
    \item Note it is \emph{not} coercive; the parties can not take the compromise.
    \item As such, it is mostly effective only if all parties are willing to act on the compromise and are actively both working to seek a resolution.
    \item Mediation can be public or private.
    \item One way they work is via interest-based mediation --- convince the parties that the compromise is in the party's best interest.
    \item The other is rights-based --- advise a party on their legal rights/status/position and give advice on what a party should do.
    \item Often these overlap; to try to convince both parties to compromise.
    \item In Canada, divorce laws require mediation to be mentioned by lawyers, to try to avoid getting courts to deal with things like property.  Mediation is also often faster and less emotionally and monetarily draining.
    \item In Ontario, you're required to use a mediator first before going to small claims court; but again, you cannot be forced.
    \item In mediation, both sides can win something.
    \item Arbitration is stronger; it is a neutral third party who has the power to settle disputes.
    \item The arbitrator can make a legally binding decision.
    \item Parties must agree that the arbitrator's decision is binding and must be accepted regardless of the outcome.
    \item Arbitration is more likely to be win-lose; someone will come out the loser.  It can still come to a compromise, of course.
    \item Finally, adjudication is like the court system.  Final, public and formal method of conflict resolution.
    \item Deals with the law, not fairness.  Like arbitration, they may seek a compromise, but a court's decision is often a zero-sum game with a winner and a loser.
\end{itemize}

\section{Administrative/Regulatory Law}
\begin{itemize}
    \item 3 main branches of governments:
        \begin{itemize}
            \item Legislative (federal and provincial), creates statutory law
            \item Judicial branch (courts), hears cases
            \item Executive branch (enforcement)
        \end{itemize}
    \item Government departments and agencies will often have the authority to develop and enforce policies in the specific areas of authority; the rules they develop and enforce are often referred to as administrative law.
    \item They can thus manage law regarding these specific areas of interest.
    \item Administrative tribunals deal with conflicts relating to administrative or regulatory statues.
    \item AKA boards, commissions, agencies, councils, etc.
    \item For example, Landlord/Tenant Board, Health and Safety Council, Disciplinary Committee, etc.
    \item They have authority over these (narrow) areas of law; for example, a psychiatric review board has power over decisions of people who are in facilities and whether they can be treated against their will.
    \item Often can set their own standards and regulations, as well as having significant powers (fines, send people to prison, keep people in a mental hospital, decertify/revoke licenses, fired, deport people, etc.).
    \item Handle conflicts between people and the government \emph{and} between citizens (ie: tenant and landlord).
    \item Tribunals are supposed to be an impartial group that should offer fair hearings for disputes.
    \item They should have some level of autonomy from higher-ups.
    \item Members are subject to conflicts of interest, as they are representative of their tribunals.
    \item For example, police tribunals used to be made of police officers, which seemed biased and unfair.
    \item They cannot act arbitrarily; they must act in ways that are relevant to the rules/regulations given their mandate.
    \item That is, they must work within their areas of jurisdiction and within law.
    \item For example, the parole board cannot deny a lawyer and must give the reasons why a parole is denied.
    \item And of course, the courts are still there; this ensures tribunals are doing their job in an impartial and fair way.
    \item That is, a citizen can go to the courts if they feel that the tribunal was unfair or not agreeing with the law.
    \item For example, some cases where patients were held in a mental hospital have gone outside and used the court and won there.
    \item One of the main advantages is that tribunals are often fast, open, and more efficient cost-wise and time-wise than a court.
    \item Tribunal members are often appointed due to expertise and knowledge; this usually means that decisions done are often going to be done by people who know what they are doing.
    \item They are also usually less formal, allowing for more issues and things brought in/discussed that courts would often not even consider.
    \item Tribunals are also supposed to be fair and impartial, just like a court.  This means that people are actually willing to use tribunals!
\end{itemize}

\end{document}
