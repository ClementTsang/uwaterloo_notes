\documentclass{article}

\usepackage{times}
\usepackage{textcomp}
\usepackage{listings}
\usepackage{fullpage}
\usepackage{color}
\usepackage{courier}
\usepackage{verbatim}
\usepackage{graphicx}
\usepackage{amsmath, amsfonts, amssymb, amsthm}
\usepackage{hyperref}
\graphicspath{{./}}

\lstset{language=python, keywordstyle={\bfseries \color{green}}, basicstyle=\footnotesize\ttfamily}
\setlength{\paperheight}{11in}
\author{Clement Tsang}

\begin{document}

\begin{center}
    \Large{LS 101 --- Chapter 4: Due Process and Crime Control Models of the Criminal Process}
\end{center}

\section{Crime Control Model}
\begin{itemize}
    \item This model argues that crime and social disorder as a major threat to our freedom.
    \item Repressing crime is the most important function of the criminal process.
    \item Maintaining social order is an important part of governance.
    \item Society willingly gives up some freedom for order.
    \item Some people will see that this is worth it; ends justify the means.
    \item The police play an important role and are the good people.
    \item The police operate on a presumption of guilt.
    \item Laws are meant to be flexible to enable the police to apprehend criminals.
    \item The model emphasizes the ends over the means.
    \item Also sides with the victims --- allows the victims to get their day in court and go after the offender.
    \item Often faster.
    \item Criticism is that it allows the police to abuse the law.
    \item An example of this being used in Canada was the Quebec Crisis when the War Measures Act was invoked.
    \item An example of a society that follows this model is China.
        \begin{itemize}
            \item China is regarded as an authoritarian country with few political freedoms.
            \item Shousen laws allow the police to detain Chinese citizens without charge.
            \item Social order is maintained through coercion.
            \item Yet, one could argue that these administrative detentions do help maintain social order.
            \item However, critics point out that the cost is too high for what it does when it comes to civil liberties.
        \end{itemize}
\end{itemize}

\section{The Due Process Model}
\begin{itemize}
    \item This is a model that assumes that the greatest threat to our liberty is from the state.
    \item Almost the opposite of the crime control model --- this is suspicious of authority, and aims to prevent the misuse of power by officials.
    \item ``Better many guilty people go free than an innocent be convicted.''
    \item Powers given to law enforcement must be limited/scrutinized.
    \item Time-consuming to maintain this standard.
    \item Focusing on the means rather than the ends.
    \item Convicting criminals is not important without the rights.
    \item The courts/judiciary have the power, including power to control the police.
    \item Evidence that violates the law is thrown out.
\end{itemize}

\section{Summary}
\begin{itemize}
    \item Both models have their benefits and deficiencies.
    \item Society must find a balance between maintaining social orders and protecting individual rights and civil liberties.
\end{itemize}

\end{document}
