\documentclass{article}

\usepackage{times}
\usepackage{textcomp}
\usepackage{listings}
\usepackage{fullpage}
\usepackage{color}
\usepackage{courier}
\usepackage{verbatim}
\usepackage{graphicx}
\usepackage{amsmath, amsfonts, amssymb, amsthm}
\usepackage{hyperref}
\graphicspath{{./}}

\lstset{language=python, keywordstyle={\bfseries \color{green}}, basicstyle=\footnotesize\ttfamily}
\setlength{\paperheight}{11in}
\author{Clement Tsang}

\begin{document}

\begin{center}
    \Large{LS 101 --- Chapter 10: Family Law and Social Policy --- Marriage and Divorce}
\end{center}

\section{Sacred, Social, and Personal Concepts of Marriage}
\begin{itemize}
    \item The sacred concept sees marriage as a religious holy union.
    \item The contract is a contract the couple makes with some higher order.
    \item The authority resides with, for example, in Christianity, God and the Church.
    \item The social concept gives priority to family obligations, and emphasizes self-sacrifice.
    \item The authority here resides with kinship, parents, and elders.
    \item In the former two, divorce is stigmatized.
    \item The personal concept gives priority to individual rights and personal happiness.
    \item Laws regulating marriage and the family are largely a provincial responsibility.  Divorce laws are federal statutes.
\end{itemize}

\section{Defining Marriage}
\begin{itemize}
    \item Marriage is a socially legitimate sexual union that is public, undertaken with some idea of permanence, and involves mutual rights and obligations.
\end{itemize}

\end{document}
