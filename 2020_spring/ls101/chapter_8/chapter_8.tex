\documentclass{article}

\usepackage{times}
\usepackage{textcomp}
\usepackage{listings}
\usepackage{fullpage}
\usepackage{color}
\usepackage{courier}
\usepackage{verbatim}
\usepackage{graphicx}
\usepackage{amsmath, amsfonts, amssymb, amsthm}
\usepackage{hyperref}
\graphicspath{{./}}

\lstset{language=python, keywordstyle={\bfseries \color{green}}, basicstyle=\footnotesize\ttfamily}
\setlength{\paperheight}{11in}
\author{Clement Tsang}

\begin{document}

\begin{center}
    \Large{LS 101 --- Chapter 8: Principles of Criminal Law}
\end{center}

\section{The Criminal Law and the \emph{Criminal Code Of Canada}}
\begin{itemize}
    \item The Criminal Code is a federal statute and applied throughout Canada.
    \item Other criminal statues include the Youth Criminal Justice Act and the Controlled Drugs and Substances Act.
    \item Most categories have to be done within Canada to be considered, though some can be done outside of Canada (though they're shit like conspiracy and genocide).
    \item There are two categories of criminal offences:
        \begin{itemize}
            \item Indictable Conviction (more serious, sentences from maximum of two years to life)
            \item Summary Conviction (minor sentences, maximum sentence of up to six months)
        \end{itemize}
    \item In an indictable offence, there may be a preliminary hearing.  The Crown has to show the court that there is enough evidence to proceed with a trial.
    \item The defence don't have to raise any evidence at all during this hearing.  They can just see what the Crown's case is.
    \item In essence, it's a trial before a trial.
    \item Comparing to the US, the US has much harsher punishments, with many \emph{minimum} sentences compared to Canada, which only has them for a certain few crimes.
\end{itemize}

\section{Offences}
\begin{itemize}
    \item Offences against the state include:
        \begin{itemize}
            \item Treason (incredibly serious, must be reported if knowledge is known about it, used to be capital; this shit is like trying to kill the Queen or waging war against Canada)
            \item Terrorism (newer, after 9/11.  Serious, up to life imprisonment)
            \item Perjury (lying under oath, can be punished for up to 10 years)
            \item Fabricating evidence
        \end{itemize}
    \item Offences against public order include:
        \begin{itemize}
            \item Committing indecent acts
            \item Causing a disturbance
            \item Mischief
        \end{itemize}
    \item Public mischief includes (a summary offence):
        \begin{itemize}
            \item Bomb scares
            \item Misleading authorities
        \end{itemize}
    \item Offences against the person include:
        \begin{itemize}
            \item Assault
            \item Manslaughter
            \item Murder
        \end{itemize}
    \item Offences against property include:
        \begin{itemize}
            \item Theft
            \item Arson
            \item Breaking and entering
            \item Embezzlement
        \end{itemize}
    \item Attempts to commit a criminal act, or being accessories to a criminal offence, can lead to being charged.
\end{itemize}

\section{Actus Reus and Mens Rea}
\begin{itemize}
    \item As mentioned before, this stands for the act and the intent.
    \item Note that an act can count as an omission (ie: having a legal duty to act).
    \item A person can only be committed if they have both the act and intent simultaneously.
    \item The Crown wants to prove guilt beyond a reasonable doubt, and show both the act and the intent to the act.
    \item That is, the Crown has to prove the person did the act and the defendant's behaviour caused the consequences coming from the act.
    \item Criminal acts must also be voluntary in order to be guilty --- if a person is forced to kill someone else they aren't guilty.
    \item The act has to be something they're physically capable of doing and that they wanted to do.  For example, if you accidentally had a spasm, or you accidentally pushed someone, these are considered as involuntary.
    \item The courts will presume that people intend the natural consequences of their actions --- for example, if you ``shoot someone to just wound them, not kill them'', the courts would not accept that.
    \item Intentionality refers to a high degree of malice and premediation.
    \item Recklessness is a lesser form of intent and refers to total indifference to the consequences of an act.
    \item Reckless actions causing death are likely to lead to a charge of second degree murder.
    \item Criminal negligence is when the accused wasn't thinking when their duty compelled them to.  It implies inadvertence.
\end{itemize}

\subsection{Offences of Absolute and Strict Liability}
\begin{itemize}
    \item Administrative/regulatory statues often consist of offences of absolute or strict liability.
    \item Offences of absolute or strict liability only require that the Crown prove that the person committed the act.
    \item There are no defences to offences of absolute liability.
    \item There are restricted defences to offences of strict liability such as due diligence.
    \item These types of offences relate to public welfare statues and allow for expediency in terms of enforcement as most people will plead guilty.
    \item These are created for public welfare.
    \item Usually little stigma and penalties are usually fine.
\end{itemize}

\end{document}
