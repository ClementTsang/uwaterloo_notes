\documentclass{article}

\usepackage{times}
\usepackage{textcomp}
\usepackage{listings}
\usepackage{fullpage}
\usepackage{color}
\usepackage{courier}
\usepackage{verbatim}
\usepackage{graphicx}
\usepackage{amsmath, amsfonts, amssymb, amsthm}
\usepackage{hyperref}
\graphicspath{{./}}

\lstset{language=python, keywordstyle={\bfseries \color{green}}, basicstyle=\footnotesize\ttfamily}
\setlength{\paperheight}{11in}
\author{Clement Tsang}

\begin{document}

\begin{center}
    \Large{LS 101 --- Chapter 9: Mental Disorder and the Law} 
\end{center}

\section{The Health Care Consent Act}
\begin{itemize}
    \item No treatment without consent unless they are incapable of giving consent (like if you're knocked out it's fine to assume).
    \item If someone is capable of giving consent in their behalf then that is the alternative for one to take.
    \item There are some elements required for consent to treatment:
        \begin{itemize}
            \item Consent must relate to treatment.
            \item Consent must be informed.
            \item Consent must be given voluntarily.
            \item Cannot be obtained by misrepresentation/fraud.
        \end{itemize}
    \item The Consent and Capacity Board will conduct hearings under the \emph{Health Care and Consent Act}, the \emph{Mental Health Act}, and the \emph{Substitute Decision Act}
    \item This board will deal with issues with consent to treatment, involuntary status of a patient in a hospital, substitute consent people not respecting the wishes of the patient, etc.
    \item Their decisions can be appealed, like many other tribunals.
\end{itemize}

\section{Mental Health Act}
\begin{itemize}
    \item The \emph{Mental Health Act of Ontario} from 1990 deals with rights of patients who are under the observation, care, and treatment in a psychiatry facility.
    \item Deals with the following sections:
        \begin{itemize}
            \item Where admission may be refused
            \item Admission of informal or voluntary patients
            \item Informal or voluntary patients
            \item Application for psychiatric assessment
            \item Authority of application
            \item Justice of the peace's order for psychiatric examination
            \item Action by police officers
            \item Change from informal or voluntary patient to involuntary patient
            \item Duty of attending physician
            \item Authority of certificate
            \item Change of status, where period of detention has expired
            \item Judge's order for examination
            \item Judge's order for admission
            \item Communications to and from patients
            \item Where communication may be withheld
        \end{itemize}
\end{itemize}

\section{Criminal Code of Canada --- The Defence of Mental Disorder}
\begin{itemize}
    \item The defence of mental disorder aims at the \emph{mens rea} component of an offence by arguing that the person did not have the capabilities to form the intent when the act was committed due to mental disorders at the time.
    \item Within Section 16 of the \emph{Criminal Code of Canada}, the defence of mental disorder can be found.
    \item It states that:
        \begin{itemize}
            \item People are not criminally responsible for an act committed, or an omission, while suffering from a mental disorder that makes them incapable of understanding what they were doing or whether it was right or wrong.
            \item People are presumed to not suffer from a mental disorder by default, unless proven.
            \item The burden of proof of the accused suffering from the mental disorder is on the party that raises the issue.
        \end{itemize}
    \item Generally, if this defence is raised, testimony is given by psychiatrists or psychologists.  They must convince the court that the mental disorder was serious enough to render the accused into committing the act without intent.
    \item Countering the defence of mental disorder will involve the Crown/defence attorney calling their own psychiatrists to testify.
    \item For a person to detain an accused in a psychiatric facility, they must pose a serious threat of harming someone and the threat must be of criminal nature.
    \item NCR persons are not meant to serve punishments.
    \item NCR (not criminally responsible) is short for NCRMD (not criminally responsible because of mental disorder).
\end{itemize}

\section{Fitness to Stand Trial}
\begin{itemize}
    \item This section of the criminal code is used when an accused is judged to be unable to understand court proceedings and assist in their defence.
    \item People are assumed to be fit to stand trial unless the court is convinced otherwise.
    \item A verdict of being unfit to stand trial does not stop the accused from being tried subsequently if they become fit to stand trial.
    \item Prim facie case to be made every two years --- the court shall hold an inquiry at no later than 2 years, and every 2 years after until the accused is acquitted or tried, to decide whether sufficient evidence can be found that shows the accused can be put on trial.
\end{itemize}

\end{document}
